\documentclass[onecolumn, amsmath, amsfonts, amssymb]{aastex62}
\usepackage{mathtools}
\usepackage{natbib}
\usepackage{bm}
\newcommand{\vdag}{(v)^\dagger}
\newcommand\aastex{AAS\TeX}
\newcommand\latex{La\TeX}


\newcommand{\Div}[1]{\ensuremath{\nabla\cdot\left( #1\right)}}
\newcommand{\DivU}{\ensuremath{\nabla\cdot\bm{u}}}
\newcommand{\angles}[1]{\ensuremath{\left\langle #1 \right\rangle}}
\newcommand{\KS}[1]{\ensuremath{\text{KS}(#1)}}
\newcommand{\KSstat}[1]{\ensuremath{\overline{\text{KS}(#1)}}}
\newcommand{\grad}{\ensuremath{\nabla}}
\newcommand{\RB}{Rayleigh-B\'{e}nard }
\newcommand{\stressT}{\ensuremath{\bm{\bar{\bar{\Pi}}}}}
\newcommand{\lilstressT}{\ensuremath{\bm{\bar{\bar{\sigma}}}}}
\newcommand{\nrho}{\ensuremath{n_{\rho}}}
\newcommand{\approptoinn}[2]{\mathrel{\vcenter{
	\offinterlineskip\halign{\hfil$##$\cr
	#1\propto\cr\noalign{\kern2pt}#1\sim\cr\noalign{\kern-2pt}}}}}

\newcommand{\appropto}{\mathpalette\approptoinn\relax}
\newcommand{\mR}{\ensuremath{\mathcal{R}}}
\newcommand{\mP}{\ensuremath{\mathcal{P}}}
\newcommand{\bu}{\ensuremath{\bm{u}}}
\newcommand{\bV}{\ensuremath{\bm{\omega}}}
\newcommand{\cross}[2]{\ensuremath{#1 \times #2}}
\newcommand{\dotp}[2]{\ensuremath{#1 \cdot #2}}
\newcommand{\curl}[1]{\ensuremath{\cross{\grad}{\left(#1\right)}}}
\newcommand{\pderiv}[2]{\ensuremath{\frac{\partial #1}{\partial #2}}}
\newcommand{\xHe}{\ensuremath{X_{\text{He}}}}

\renewcommand{\bar}[1]{\ensuremath{\overline{#1}}}
\renewcommand{\vec}[1]{\ensuremath{\mathbf{#1}}}
\renewcommand{\dot}{\ensuremath{\cdot}}

\usepackage{color}
\newcommand{\gv}[1]{{\color{blue} #1}}

\begin{document}
\section{Parameters and some definitions}
We're going to study penetrative convection in a compressible ideal gas.
To start with, let's make some definitions.
\begin{enumerate}
    \item The Brunt-Vaisaila frequency is $N^2 = g \grad s / c_P$, where $g$ is the gravitational acceleration, $\grad s$ is the entropy gradient, and $c_p$ is the specific heat at constant pressure.
    \item the convective frequency is $f_{\rm{conv}} = 1/\tau_{\rm{conv}}$, where $\tau_{\rm{conv}}$ is a characteristic convective timescale.
    \item We will assume the characteristic convective timescale is the timescale of internal heating, $\tau_{\rm{conv}} = (\rho L^2 / Q)^{1/3}$, where $Q$ is a heating term with units [energy / time / volume], $\rho$ is the density, and $L$ is the characteristic convective length scale.
    \item The system will be heated at a constant rate in a layer of depth $\delta_H$ so that the flux carried by convection is $F_{\rm{conv}} = Q \delta_H$.
    \item The mean radiative conductivity varies from a small value $\kappa_{\rm{CZ}}$ in the convection zone to a large vale $\kappa_{\rm{RZ}}$ in the radiative zone.
    \item Since there is some flux $F_{\rm{bound}} = - \kappa_{\rm{CZ}}\grad T_{\rm{ad}}$ conducted along the convective boundary, there is a total flux $F_{\rm{tot}} = F_{\rm{conv}} + F_{\rm{bound}}$ carried in the system.
        We will define a parameter $\mu = F_{\rm{bound}}/F_{\rm{conv}}$ so that $F_{\rm{tot}} = (1 + \mu) F_{\rm{conv}}$.
    \item In the radiative zone, all of the flux is carried by radiation, $F_{\rm{rad}}^{\rm{RZ}} = F_{\rm{tot}}$.
    \item In the hypothetical adiabatic penetrative zone, the flux is $F_{\rm{tot}} = F_{\rm{rad}}^{\rm{PZ}} + F_{\rm{conv}}^{\rm{PZ}}$.
        We define the penetration parameter:
        \begin{equation}
            \mathcal{P} \equiv -\frac{F_{\rm{conv}}}{F_{\rm{conv}}^{\rm{PZ}}}.
        \end{equation}
        Furthermore we note that $F_{\rm{rad}}^{\rm{PZ}} = F_{\rm{rad}}^{\rm{RZ}}(\grad T_{\rm{ad}} / \grad T_{\rm{RZ}})$.
        Since $F_{\rm{rad}}^{\rm{RZ}} = F_{\rm{tot}}$, we get
        \begin{equation}
            F_{\rm{tot}}\left(1 - \frac{\grad T_{\rm{ad}}}{\grad T_{\rm{RZ}}}\right)
            = - \frac{F_{\rm{conv}}^{\rm{CZ}}}{\mathcal{P}}.
        \end{equation}
        Rearranging, we can define $\mP$ as
        \begin{equation}
            \boxed{
                \mathcal{P} = \left[(1 + \mu) \left(\frac{\grad T_{\rm{ad}}}{\grad T_{\rm{RZ}}} - 1\right)\right]^{-1}}.
        \end{equation}
        We expect the size of a hypothetical penetration zone to get large when this gets large, and vice versa.
        Note that it will also be useful to express the ratio of the temperature gradients here:
        \begin{equation}
            \frac{\grad T_{\rm{ad}}}{\grad T_{\rm{RZ}}} = 1 + [\mathcal{P}(1 + \mu)]^{-1},
        \end{equation}
        so $|\grad T_{\rm{ad}}| > |\grad T_{\rm{RZ}}|$.
    \item It is useful to define the stiffness,
        \begin{equation}
            \mathcal{S} \equiv \frac{N_{\rm{RZ}}^2}{f_{\rm{conv}}}^2
            = \left(\frac{\rho L^2}{Q}\right)^{2/3}\left(\frac{g \grad s_{\rm{RZ}}}{c_P}\right).
        \end{equation}
        From the ideal gas equation of state, we know
        \begin{equation}
            \frac{\grad s_{\rm{RZ}}}{c_P} = \grad\left[\frac{1}{\gamma}\ln T - \frac{\gamma-1}{\gamma}\ln\rho\right]
            \rightarrow
            \frac{1}{T}\left[\grad T - \grad T_{\rm{ad}}\right],
        \end{equation}
        where we have assumed that hydrostatic equilibrium applies, $T\grad \ln \rho = -g\hat{z} / R - \grad T$ and $\grad T_{\rm{ad}} = -g/c_P\hat{z}$ and $c_P = R \gamma/(\gamma-1)$.
        So
        \begin{equation}
            S = \left(\frac{\rho L^2}{Q}\right)^{2/3}\frac{g }{T}\left(\grad T_{\rm{RZ}} - \grad T_{\rm{ad}}\right)
                = \left(\frac{\rho L^2}{Q}\right)^{2/3}\frac{g \grad T_{\rm{ad}}}{T}\left(\frac{1}{1 + [\mathcal{P}(1 + \mu)]^{-1}} - 1\right).
        \end{equation}
        Defining the adiabatic gradient $\grad_{\rm{ad}} = (d\ln T / d\ln P)|_{\mathcal{S}}$, and then $\grad T_{\rm{ad}} = \grad_{\rm{ad}} h^{-1} T$ where $h = (d\ln P/d\ln z)^{-1}$ is the pressure scale height, we get
        \begin{equation}
            S = \grad_{\rm{ad}}\frac{g \tau_{\rm{conv}}^2}{h}\left(\frac{1}{1 + [\mathcal{P}(1 + \mu)]^{-1}} - 1\right).
        \end{equation}
        In hydrostatic equilibrium, $g h = P/\rho = c_s^2 / \gamma$, and if we define the convective velocity $u_{\rm{conv}} = L/\tau_{\rm{conv}}$, we find
        \begin{equation}
            \frac{g \tau^2}{h} = \frac{1}{\gamma}\frac{c_s^2}{u_{\rm{conv}}^2}\left(\frac{L}{h}\right)^2.
        \end{equation}
        Defining the mach number of convection $\mathcal{M} = u_{\rm{conv}}/c_s$, we find
        \begin{equation}
            \boxed{
                S = \mathcal{M}^{-2}\frac{\grad_{\rm{ad}}}{\gamma}\left(\frac{L}{h}\right)^2\left(\frac{1}{1 + [\mathcal{P}(1 + \mu)]^{-1}} - 1\right)}.
        \end{equation}
\end{enumerate}

\end{document}
